\section{Ход работы}

1. Используя  команды ECHO, PRINTF вывести  информационные сообщения на экран

Текст скрипта с комментариями последовательности действий представлен на рисунке \ref{task1_lst}.

\addimg{1_1}{1}{Текст скрипта для задания 1}{task1_lst}

В результате запуска скрипта с помощью команды sh, получим результат, представленный на рисунке \ref{task1_res}.

\addimg{1_2}{1}{Результат выполнения скрипта для задания 1}{task1_res}

2. Присвоить  переменной  А  целочисленное  значение.  Просмотреть значение переменной А

Текст скрипта с комментариями последовательности действий представлен на рисунке \ref{task2_lst}.

\addimg{2_1}{1}{Текст скрипта для задания 2}{task2_lst}

В результате запуска скрипта с помощью команды sh, получим результат, представленный на рисунке \ref{task2_res}.

\addimg{2_2}{1}{Результат выполнения скрипта для задания 2}{task2_res}

3. Присвоить  переменной  В  значение  переменной  А.  Просмотреть значение переменной В

Текст скрипта с комментариями последовательности действий представлен на рисунке \ref{task3_lst}.

\addimg{3_1}{0.8}{Текст скрипта для задания 3}{task3_lst}

В результате запуска скрипта с помощью команды sh, получим результат, представленный на рисунке \ref{task3_res}.

\addimg{3_2}{1}{Результат выполнения скрипта для задания 3}{task3_res}

4. Присвоить переменной С значение "путь до своего каталога". Перейти в этот каталог с использованием переменной.

Текст скрипта с комментариями последовательности действий представлен на рисунке \ref{task4_lst}.

\addimg{4_1}{1}{Текст скрипта для задания 4}{task4_lst}

Перейдем в корневой каталог с помощью команды cd /, затем запустим скрипт с помощью команды ., чтобы процесс скрипта запустился в текущем экземпляре bash. В результате запуска скрипта с помощью команды ., получим результат, представленный на рисунке \ref{task4_res}.

\addimg{4_2}{1}{Результат выполнения скрипта для задания 4}{task4_res}

5. Присвоить переменной Dзначение “имя команды”, а именно, команды DATE. Выполнить эту команду, используя значение переменной

Текст скрипта с комментариями последовательности действий представлен на рисунке \ref{task5_lst}.

\addimg{5_1}{1}{Текст скрипта для задания 3}{task5_lst}

В результате запуска скрипта с помощью команды sh, получим результат, представленный на рисунке \ref{task5_res}.

\addimg{5_2}{1}{Результат выполнения скрипта для задания 5}{task5_res}

6. Присвоить переменной E значение “имя команды”, а именно, команды просмотра   содержимого   файла,   просмотреть   содержимое   переменной. Выполнить эту команду, используя значение переменной

Текст скрипта с комментариями последовательности действий представлен на рисунке \ref{task6_lst}.

\addimg{6_1}{1}{Текст скрипта для задания 6}{task6_lst}

В результате запуска скрипта с помощью команды sh, получим результат, представленный на рисунке \ref{task6_res}.

\addimg{6_2}{1}{Результат выполнения скрипта для задания 6}{task6_res}

7.  Присвоить  переменной F значение  “имя  команды”,  а  именно сортировки содержимого текстового файла. Выполнить эту команду, используя значение переменной

Текст скрипта с комментариями последовательности действий представлен на рисунке \ref{task7_lst}.

\addimg{7_1}{1}{Текст скрипта для задания 7}{task7_lst}

В результате запуска скрипта с помощью команды sh, получим результат, представленный на рисунке \ref{task7_res}.

\addimg{7_2}{1}{Результат выполнения скрипта для задания 7}{task7_res}

8.Программа  запрашивает  значение  переменной,  а  затем  выводит значение этой переменной

Текст скрипта с комментариями последовательности действий представлен на рисунке \ref{task8_lst}.

\addimg{8_1}{0.8}{Текст скрипта для задания 8}{task8_lst}

В результате запуска скрипта с помощью команды sh, получим результат, представленный на рисунке \ref{task8_res}.

\addimg{8_2}{1}{Результат выполнения скрипта для задания 8}{task8_res}

9. Программа  запрашивает  имя  пользователя,  затем  здоровается  с  ним, используя значение введенной переменной

Текст скрипта с комментариями последовательности действий представлен на рисунке \ref{task9_lst}.

\addimg{9_1}{1}{Текст скрипта для задания 9}{task9_lst}

В результате запуска скрипта с помощью команды sh, получим результат, представленный на рисунке \ref{task9_res}.

\addimg{9_2}{1}{Результат выполнения скрипта для задания 9}{task9_res}

10. Программа запрашивает значения двух переменных, вычисляет сумму(разность,  произведение,  деление)  этих  переменных.  Результат  выводится  на экран (использовать команды a) EXPR; б) ВС)

Текст скрипта с комментариями последовательности действий представлен на рисунке \ref{task10_lst}.

\addimg{10_1}{1}{Текст скрипта для задания 10}{task10_lst}

В результате запуска скрипта с помощью команды sh, получим результат, представленный на рисунке \ref{task10_res}.

\addimg{10_2}{1}{Результат выполнения скрипта для задания 10}{task10_res}

