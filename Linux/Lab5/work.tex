\section{Ход работы}

1. Используя  команды ECHO, PRINTF вывести  информационные сообщения на экран

Текст скрипта с комментариями последовательности действий представлен на рисунке \ref{task1_lst}.

\addimg{1_1}{1}{Текст скрипта для задания 1}{task1_lst}

В результате запуска скрипта с помощью команды sh, получим результат, представленный на рисунке \ref{task1_res}.

\addimg{1_2}{1}{Результат выполнения скрипта для задания 1}{task1_res}

2. Присвоить  переменной  А  целочисленное  значение.  Просмотреть значение переменной А

Текст скрипта с комментариями последовательности действий представлен на рисунке \ref{task2_lst}.

\addimg{2_1}{1}{Текст скрипта для задания 2}{task2_lst}

В результате запуска скрипта с помощью команды sh, получим результат, представленный на рисунке \ref{task2_res}.

\addimg{2_2}{1}{Результат выполнения скрипта для задания 2}{task2_res}

3. Присвоить  переменной  В  значение  переменной  А.  Просмотреть значение переменной В

Текст скрипта с комментариями последовательности действий представлен на рисунке \ref{task3_lst}.

\addimg{3_1}{0.8}{Текст скрипта для задания 3}{task3_lst}

В результате запуска скрипта с помощью команды sh, получим результат, представленный на рисунке \ref{task3_res}.

\addimg{3_2}{1}{Результат выполнения скрипта для задания 3}{task3_res}

4. Присвоить переменной С значение "путь до своего каталога". Перейти в этот каталог с использованием переменной.

Текст скрипта с комментариями последовательности действий представлен на рисунке \ref{task4_lst}.

\addimg{4_1}{1}{Текст скрипта для задания 4}{task4_lst}

Перейдем в корневой каталог с помощью команды cd /, затем запустим скрипт с помощью команды ., чтобы процесс скрипта запустился в текущем экземпляре bash. В результате запуска скрипта с помощью команды ., получим результат, представленный на рисунке \ref{task4_res}.

\addimg{4_2}{1}{Результат выполнения скрипта для задания 4}{task4_res}

5. Присвоить переменной Dзначение “имя команды”, а именно, команды DATE. Выполнить эту команду, используя значение переменной

Текст скрипта с комментариями последовательности действий представлен на рисунке \ref{task5_lst}.

\addimg{5_1}{1}{Текст скрипта для задания 3}{task5_lst}

В результате запуска скрипта с помощью команды sh, получим результат, представленный на рисунке \ref{task5_res}.

\addimg{5_2}{1}{Результат выполнения скрипта для задания 5}{task5_res}

6. Присвоить переменной E значение “имя команды”, а именно, команды просмотра   содержимого   файла,   просмотреть   содержимое   переменной. Выполнить эту команду, используя значение переменной

Текст скрипта с комментариями последовательности действий представлен на рисунке \ref{task6_lst}.

\addimg{6_1}{1}{Текст скрипта для задания 6}{task6_lst}

В результате запуска скрипта с помощью команды sh, получим результат, представленный на рисунке \ref{task6_res}.

\addimg{6_2}{1}{Результат выполнения скрипта для задания 6}{task6_res}

7.  Присвоить  переменной F значение  “имя  команды”,  а  именно сортировки содержимого текстового файла. Выполнить эту команду, используя значение переменной

Текст скрипта с комментариями последовательности действий представлен на рисунке \ref{task7_lst}.

\addimg{7_1}{1}{Текст скрипта для задания 7}{task7_lst}

В результате запуска скрипта с помощью команды sh, получим результат, представленный на рисунке \ref{task7_res}.

\addimg{7_2}{1}{Результат выполнения скрипта для задания 7}{task7_res}

8.Программа  запрашивает  значение  переменной,  а  затем  выводит значение этой переменной

Текст скрипта с комментариями последовательности действий представлен на рисунке \ref{task8_lst}.

\addimg{8_1}{0.8}{Текст скрипта для задания 8}{task8_lst}

В результате запуска скрипта с помощью команды sh, получим результат, представленный на рисунке \ref{task8_res}.

\addimg{8_2}{1}{Результат выполнения скрипта для задания 8}{task8_res}

9. Программа  запрашивает  имя  пользователя,  затем  здоровается  с  ним, используя значение введенной переменной

Текст скрипта с комментариями последовательности действий представлен на рисунке \ref{task9_lst}.

\addimg{9_1}{1}{Текст скрипта для задания 9}{task9_lst}

В результате запуска скрипта с помощью команды sh, получим результат, представленный на рисунке \ref{task9_res}.

\addimg{9_2}{1}{Результат выполнения скрипта для задания 9}{task9_res}

10. Программа запрашивает значения двух переменных, вычисляет сумму(разность,  произведение,  деление)  этих  переменных.  Результат  выводится  на экран (использовать команды a) EXPR; б) ВС)

Текст скрипта с комментариями последовательности действий представлен на рисунке \ref{task10_lst}.

\addimg{10_1}{1}{Текст скрипта для задания 10}{task10_lst}

В результате запуска скрипта с помощью команды sh, получим результат, представленный на рисунке \ref{task10_res}.

\addimg{10_2}{1}{Результат выполнения скрипта для задания 10}{task10_res}

11. Вычислить  объем  цилиндра.  Исходные  данные  запрашиваютсяпрограммой. Результат выводится на экран.

Текст скрипта с комментариями последовательности действий представлен на рисунке \ref{task11_lst}.

\addimg{11_1}{1}{Текст скрипта для задания 11}{task11_lst}

В результате запуска скрипта с помощью команды sh, получим результат, представленный на рисунке \ref{task11_res}.

\addimg{11_2}{1}{Результат выполнения скрипта для задания 11}{task11_res}

12. Используя  позиционные  параметры,  отобразить  имя  программы,количество  аргументов  командной  строки,  значение  каждого  аргумента командной строки.

Текст скрипта с комментариями последовательности действий представлен на рисунке \ref{task12_lst}.

\addimg{12_1}{1}{Текст скрипта для задания 12}{task12_lst}

В результате запуска скрипта с помощью команды sh, получим результат, представленный на рисунке \ref{task12_res}.

\addimg{12_2}{1}{Результат выполнения скрипта для задания 12}{task12_res}

13. Используя позиционный параметр, отобразить содержимое текстовогофайла, указанного в качестве аргумента командной строки. После паузы экран очищается.

Текст скрипта с комментариями последовательности действий представлен на рисунке \ref{task13_lst}.

\addimg{13_1}{1}{Текст скрипта для задания 13}{task13_lst}

В результате запуска скрипта с помощью команды sh, получим результат, представленный на рисунке \ref{task13_res}.

\addimg{13_2}{1}{Результат выполнения скрипта для задания 13}{task13_res}

14. Используя оператор FOR, отобразить содержимое текстовых файловтекущего каталога поэкранно.

Текст скрипта с комментариями последовательности действий представлен на рисунке \ref{task14_lst}.

\addimg{14_1}{1}{Текст скрипта для задания 14}{task14_lst}

В результате запуска скрипта с помощью команды sh lab5.sh top-output.txt test.log, получим результат, представленный на рисунке \ref{task14_res}.

\addimg{14_2}{1}{Результат выполнения скрипта для задания 14}{task14_res}

15. Программой  запрашивается  ввод  числа,  значение  которого  затем сравнивается с допустимым значением. В результате этого сравнения на экран выдаются соответствующие сообщения.

Текст скрипта с комментариями последовательности действий представлен на рисунке \ref{task15_lst}.

\addimg{15_1}{1}{Текст скрипта для задания 15}{task15_lst}

В результате запуска скрипта с помощью команды sh, получим результат, представленный на рисунке \ref{task15_res}.

\addimg{15_2}{1}{Результат выполнения скрипта для задания 15}{task15_res}

16. Программой  запрашивается  год,  определяется,  високосный  ли  он.Результат выдается на экран.

Текст скрипта с комментариями последовательности действий представлен на рисунке \ref{task16_lst}.

\addimg{16_1}{1}{Текст скрипта для задания 16}{task16_lst}

В результате запуска скрипта с помощью команды sh, получим результат, представленный на рисунке \ref{task16_res}.

\addimg{16_2}{1}{Результат выполнения скрипта для задания 16}{task16_res}

17. Вводятся  целочисленные  значения  двух  переменных.  Вводится диапазон данных. Пока значения переменных находятся в указанном диапазоне, их значения инкрементируются.

Текст скрипта с комментариями последовательности действий представлен на рисунке \ref{task17_lst}.

\addimg{17_1}{1}{Текст скрипта для задания 17}{task17_lst}

В результате запуска скрипта с помощью команды sh, получим результат, представленный на рисунке \ref{task17_res}.

\addimg{17_2}{1}{Результат выполнения скрипта для задания 17}{task17_res}

18. В  качестве  аргумента  командной  строки  указывается  пароль.  Если пароль  введен  верно,  постранично  отображается  в  длинном  формате  с указанием скрытых файлов содержимое каталога /etc.

Текст скрипта с комментариями последовательности действий представлен на рисунке \ref{task18_lst}.

\addimg{18_1}{1}{Текст скрипта для задания 18}{task18_lst}

В результате запуска скрипта с помощью команды sh lab5.sh password, получим результат, представленный на рисунке \ref{task18_res1}.

\addimg{18_2}{1}{Результат выполнения скрипта для задания 18}{task18_res1}

В результате запуска скрипта с помощью команды sh lab5.sh p, получим результат, представленный на рисунке \ref{task18_res2}.

\addimg{18_3}{1}{Результат выполнения скрипта для задания 18}{task18_res2}

19. Проверить,  существует  ли  файл.  Если  да,  выводится  на  экран  его содержимое, если нет - выдается соответствующее сообщение.

Текст скрипта с комментариями последовательности действий представлен на рисунке \ref{task19_lst}.

\addimg{19_1}{1}{Текст скрипта для задания 19}{task19_lst}

В результате запуска скрипта с помощью команды sh, получим результат, представленный на рисунке \ref{task19_res}.

\addimg{19_2}{1}{Результат выполнения скрипта для задания 19}{task19_res}

20. Если файл есть каталог и этот каталог можно читать, просматривается содержимое этого каталога. Если каталог отсутствует, он создается. Если файл не есть каталог, просматривается содержимое файла.

Текст скрипта с комментариями последовательности действий представлен на рисунке \ref{task20_lst}.

\addimg{20_1}{1}{Текст скрипта для задания 20}{task20_lst}

В результате запуска скрипта с помощью команды sh, получим результат, представленный на рисунке \ref{task20_res}.

\addimg{20_2}{1}{Результат выполнения скрипта для задания 20}{task20_res}

21. Анализируются  атрибуты  файла.  Если  первый  файл  существует  и используется для чтения, а второй файл существует и используется для записи, то  содержимое  первого  файла  перенаправляется  во  второй  файл.  Вслучае несовпадений указанных атрибутов или отсутствия файлов на экран выдаются соответствующие  сообщения  (использовать  а)  имена  файлов;  б)  позиционные параметры).

Текст скрипта для пункта а) с комментариями последовательности действий представлен на рисунке \ref{task21_lst_a}.

\addimg{21_1_a}{1}{Текст скрипта для задания 21 пункт а)}{task21_lst_a}

В результате запуска скрипта с помощью команды sh, получим результат, представленный на рисунке \ref{task21_res_a}.

\addimg{21_2_a}{1}{Результат выполнения скрипта для задания 21 пункт а)}{task21_res_a}

Текст скрипта для пункта б) с комментариями последовательности действий представлен на рисунке \ref{task21_lst_b}.

\addimg{21_1_b}{1}{Текст скрипта для задания 21 пункт б)}{task21_lst_b}

В результате запуска скрипта с помощью команды sh, получим результат, представленный на рисунке \ref{task21_res_a}.

\addimg{21_2_b}{1}{Результат выполнения скрипта для задания 21 пункт б)}{task21_res_b}

22. Если  файл  запуска  программы  найден,  программа  запускается  (по выбору).

Текст скрипта с комментариями последовательности действий представлен на рисунке \ref{task22_lst}.

\addimg{22_1}{1}{Текст скрипта для задания 22}{task22_lst}

В результате запуска скрипта с помощью команды sh, получим результат, представленный на рисунке \ref{task22_res}.

\addimg{22_2}{1}{Результат выполнения скрипта для задания 22}{task22_res}

23. В качестве позиционного параметра задается файл, анализируется его размер.  Если  размер  файла  больше  нуля,  содержимое  файла  сортируется  по первому столбцу по возрастанию, отсортированная информация помещается в другой файл, содержимое которого затем отображается на экране.

Текст скрипта с комментариями последовательности действий представлен на рисунке \ref{task23_lst}.

\addimg{23_1}{1}{Текст скрипта для задания 23}{task23_lst}

В результате запуска скрипта с помощью команды sh, получим результат, представленный на рисунке \ref{task23_res}.

\addimg{23_2}{1}{Результат выполнения скрипта для задания 23}{task23_res}

24. Командой TAR осуществляется  сборка  всех  текстовых  файлов текущего каталога в один архивный файл my.tar, после паузы просматривается содержимое  файла my.tar,  затем  командой GZIP архивный  файл my.tar сжимается.

Текст скрипта с комментариями последовательности действий представлен на рисунке \ref{task24_lst}.

\addimg{24_1}{1}{Текст скрипта для задания 24}{task24_lst}

В результате запуска скрипта с помощью команды sh, получим результат, представленный на рисунке \ref{task24_res}.

\addimg{24_2}{1}{Результат выполнения скрипта для задания 24}{task24_res}

25. Написать  скрипт  с  использованием  функции,  например,  функции,суммирующей значения двух переменных.

Текст скрипта с комментариями последовательности действий представлен на рисунке \ref{task25_lst}.

\addimg{25_1}{1}{Текст скрипта для задания 25}{task25_lst}

В результате запуска скрипта с помощью команды sh, получим результат, представленный на рисунке \ref{task25_res}.

\addimg{25_2}{1}{Результат выполнения скрипта для задания 25}{task25_res}








