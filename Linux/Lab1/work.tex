\section{Ход выполнения}

На компьютер была установлена вторая операционая система Ubuntu 18.04 LTS. На рисунке \ref{start} представлен рабочий стол после загрузки ОС.
\addimg{start}{1}{Запуск Linux Ubuntu}{start}

На рисунке \ref{root} продемонстрирована загрузка пользователя root с помощью команды sudo su.
\addimg{1}{0.8}{Загрузка пользователя root}{root}

На рисунке \ref{system} показана структкра системного каталога. Для вывода данной структкры была использована команда ls.
\addimg{2}{1}{Ознакомление со структурой системного каталога}{system}
Перечень каталогов, представленных на рисунке \ref{system}, и их назанчение:
\begin{enumerate}
    \item /bin -- содержит команды, которые могут использоваться как системным администратором, так и рядовыми пользователями, причем только те команды, которые необходимы, когда никакая другая файловая система еще не смонтирована (например, в одно-пользовательском режиме). В этом каталоге могут также содержаться команды, которые используются не напрямую пользователем, а через скрипты;
    \item /boot -- каталог содержит все, что необходимо в процессе загрузки, исключая конфигурационные файлы и the map installer. Таким образом, в /boot хранятся данные, которые используются до того, как ядро начинает исполнять программы пользователя. Здесь же находятся резервные сохраненные копии главной загрузочной записи (master boot sectors), sector map files, и другие данные, которые не подлежат прямому редактированию;
    \item /dev -- это место расположения специальных файлов устройств;
    \item /etc -- содержит конфигурационные файлы и каталоги, специфичные для данной конкретной системы;
    \item /home -- домашняя директория пользователей, это достаточно стандартное решение, очевидно только, что этот каталог является специфичным для каждого отдельного компьютера;
    \item /lib -- содержит те разделяемые библиотеки, которые необходимы для загрузки системы и запуска команд, расположенных в корневой файловой системе, то есть в каталогах /bin и /sbin;
    \item /lib64 --  обычно это используется для поддержки 64-битного или 32-битного формата в системах, поддерживающих несколько форматов исполняемых файлов, и требующих библиотек с одним и тем же названием. В этом случае /lib32 и /lib64 могут быть библиотечными каталогами, а /lib - символической ссылкой на один из них;
    \item /mnt -- эта директория предназначена для того, чтобы системный администратор мог временно монтировать файловые системы по мере необходимости. Содержимое этого каталога индивидуально для каждой системы и не должно никаким образом влиять на работу запускаемых программ;
    \item /opt -- зарезервирован для установки дополнительных пакетов программного обеспечения. Пакет, который устанавливается в каталог /opt, должен размещать свои статические файлы в отдельной каталоговой структуре /opt/<package>, где <package> - название соответствующего пакета программного обеспечения;
    \item /root -- домашний каталог пользователя root;
    \item /sbin -- утилиты для выполнения задач системного администрирования (и другие команды, используемые только пользователем root) размещаются в /sbin, /usr/sbin и /usr/local/sbin. Каталог /sbin содержит исполняемые файлы, необходимые для загрузки системы и ее восстановления в различных ситуациях (restoring, recovering, and/or repairing the system) и не попавшие в каталог /bin;
    \item /tmp -- каталог для хранения временных файлов программ. Каталог /tmp должен быть доступен для программ, которым необходимы временные файлы. Программы не должны предполагать, что какой-либо файл в каталоге /tmp сохранится при следующем запуске программы;
    \item /media -- этот каталог содержит подкаталоги, которые используются в качестве точек монтирования для съемных носителей, таких как гибкие диски, компакт-диски и zip-диски;
    \item /run -- этот каталог содержит данные системной информации, описывающие систему с момента ее загрузки. Файлы в этом каталоге должны быть очищены (при необходимости удалены или усечены) в начале процесса загрузки;
    \item /srv -- параметры, которые специфичные для окружения системы. Чаще всего данная директория пуста;
    \item /usr -- в этом каталоге хранятся все установленные пакеты программ, документация, исходный код ядра и система X Window. Все пользователи кроме суперпользователя root имеют доступ только для чтения. Может быть смонтирована по сети и может быть общей для нескольких машин;
    \item /var -- это каталог для часто меняющихся данных. Здесь находятся журналы операционной системы, системные log-файлы, cache-файлы и т. д.;
    \item /lost+found -- в lost+found скидываются файлы, на которых не было ссылок ни в одной директории, хотя их inod не были помечены как свободные;
    \item /proc -- это директория, к которой примонтирована виртуальная файловая система procfs. Различная информация, которую ядро может сообщить пользователям, находится в "файлах" каталога /proc;
    \item /sys -- это директория, к которой примонтирована виртуальная файловая система sysfs, которая добавляет в пространство пользователя информацию ядра Linux о присутствующих в системе устройствах и драйверах;
    \item /snap -- каталог / snap по умолчанию является местом, где файлы и папки из установленных пакетов snap появляются в вашей системе.
\end{enumerate}

\newpage
На рисунке \ref{dev} показано содержимое каталога физических устройств.
\addimg{3}{1}{Содержимое каталога /dev}{dev}

Перечень файлов физических устройств, представленных на рисунке \ref{dev}, и их назначение:
\begin{enumerate}
    \item acpi\_thermal\_rel -- обеспечивает функции управления температурой модуля ACPI;
    \item autofs -- цель autofs - обеспечить монтирование по требованию и автоматическое размонтирование других файловых систем;
    \item btrfs-control -- устройство принимает некоторые вызовы ioctl, которые могут выполнять следующие действия с модулем файловой системы: сканирование устройства на наличие файловой системы btrfs (т.е. позволить файловым системам с несколькими устройствами монтировать автоматически) и регистрировать их в модуле ядра,   аналогично сканированию, но также дождается завершения процесса сканирования устройства для данной файловой системы, получение поддерживаемые функци;
    \item console -- текстовый терминал и виртуальные консоли;
    \item cpu\_dma\_latency -- часть интерфейса качества и обслуживания в ядре Linux
    \item cuse -- символьные устройства в пространстве пользователя
    \item drm\_dp\_aux<N> -- канал DisplayPort AUX
    \item ecryptfs -- POSIX-совместимая промышленного уровня файловая система многоуровневого шифрования для Linux
    \item fb<N> -- устройство обеспечивает абстракцию для графического оборудования
    \item freefall -- это простой демон, обеспечивающий защиту жесткого диска от ударов для ноутбуков HP, поддерживающий функцию, официально называемую «HP Mobile Data Protection System 3D» или «HP 3D DriveGuard».
    \item fuse -- (filesystem in userspace — «файловая система в пользовательском пространстве») — свободный модуль для ядер Unix-подобных операционных систем, позволяет разработчикам создавать новые типы файловых систем, доступные для монтирования пользователями без привилегий (прежде всего — виртуальных файловых систем);
    \item hpet -- таймер событий высокой точности (HPET) - это аппаратный таймер, используемый в персональных компьютерах.
    \item hwrng -- генератор случайных чисел;
    \item i2c-<N> -- последовательная асимметричная шина для связи между интегральными схемами внутри электронных приборов;
    \item kmsg -- узел символьного устройства обеспечивает доступ пользователя к буферу printk ядра;
    \item kvm -- виртуальная машина на основе ядра
    \item loop<N> -- в Linux работа с образами дисков осуществляется через так называемые петлевые (loop) устройства. Образ привязывается к loop-устройству, после этого система может работать с этим устройством, как с обычным блочным;
    \item loop-control -- начиная с Linux 3.1, ядро предоставляет устройство dev / loop-control, которое позволяет приложению динамически находить свободное устройство, а также добавлять и удалять устройства loop из системы;
    \item mcelog -- серверная часть пользовательского пространства для регистрации ошибок машинных проверок, сообщаемых ядру аппаратными средствами. Ядро выполняет немедленные действия (например, завершает процессы и т. д.), а mcelog декодирует ошибки и управляет различными другими расширенными ответами на ошибки, такими как отключение памяти, процессоров или запускающих событий. Кроме того, mcelog также обрабатывает исправленные ошибки, регистрируя их;
    \item mei<N> -- это изолированный и защищенный вычислительный ресурс (сопроцессор), находящийся внутри определенных наборов микросхем Intel;
    \item mem -- это файл символьного устройства, представляющий собой образ основной памяти компьютера. Его можно использовать, например, для проверки (и даже исправления) системы;
    \item null -- специальный файл в системах класса UNIX, представляющий собой так называемое «пустое устройство». Запись в него происходит успешно, независимо от объёма «записанной» информации. Чтение из /dev/null эквивалентно считыванию конца файла (EOF);
    \item nvram -- обеспечивает доступ к конфигурации BIOS NVRAM в системах i386 и amd64;
    \item port -- символьное устройство для чтения и / или записи;
    \item ppp -- обеспечивает реализацию функциональных возможностей, которые используются в любой реализации PPP, включая: блок сетевого интерфейса (ppp0 и т. д.), интерфейс к сетевому коду, многоканальный PPP: разделение дейтаграмм между несколькими ссылками, а также упорядочивание и объединение полученных фрагментов, интерфейс к pppd, через символьное устройство / dev / ppp, сжатие и распаковка пакетов, сжатие и распаковка заголовков TCP / IP, обнаружение сетевого трафика для набора по требованию и для тайм-аутов простоя, простая фильтрация пакетов;
    \item psaux -- устройство мыши PS / 2;
    \item ptmx -- используется для создания пары псевдотерминалов ведущего и ведомого;
    \item random -- предоставляет интерфейс к системному генератору случайных чисел, который выводит шум из драйверов устройств и других источников в «хаотичный» пул;
    \item rfkill -- предоставляет общий интерфейс для отключения любого радиопередатчика в системе;
    \item rtc<N> -- часы реального времени;
    \item sda -- первый жесткий диск;
    \item sda<N> -- N-ый раздел первого жесткого диска;
    \item sdb -- второй жесткий диск;
    \item sdb<N> -- N-ый раздел второго жесткого диска;
    \item sg<N> -- SCSI Generic driver используется, среди прочего, для сканеров, устройств записи компакт-дисков и чтения аудио-компакт-дисков в цифровом формате;
    \item snapshot -- поддержка снимков устройства;
    \item tmp<N> -- разрешает доступ к устройству Trusted Platform Module (tpm);
    \item tty<N> -- виртуальная консоль;
    \item ttyprintk -- драйвер псевдо TTY, который позволяет пользователям создавать сообщения printk через вывод на устройство ttyprintk;
    \item uhid -- поддержка драйвера ввода-вывода пользовательского пространства для подсистемы HID;
    \item uinput -- поддержка драйвера уровня пользователя для ввода;
    \item urandom -- более быстрая и менее безопасная генерация случайных чисел;
    \item userio -- призван упростить жизнь разработчикам драйверов ввода, позволяя им тестировать различные устройства Serio (в основном, различные сенсорные панели на ноутбуках), не имея физического устройства перед ними;
    \item vcs<N> -- текущее текстовое содержимое виртуальной консоли;
    \item vcsa<N> -- текущее содержимое текстового атрибута виртуальной консоли;
    \item vcsu<N> -- текущее текстовое содержимое виртуальной консоли(юникод);
    \item vga\_arbiter -- сканирует все устройства PCI и добавляет в арбитраж VGA. Затем арбитр включает / отключает декодирование на разных устройствах устаревших инструкций VGA;
    \item vhci -- виртуальный драйвер HCI Bluetooth;
    \item vhost-net -- ускоритель ядра хоста для virtio ne;
    \item vhost-vsock -- программное устройство, поэтому нет пробного вызова, который вызывает драйвер, чтобы зарегистрировать его узел устройства misc char. Это создает проблема с курицей и яйцом: приложения в пользовательском пространстве должны открываться/ dev / vhost-vsock, чтобы использовать драйвер, но файл не существует, пока модуль ядра загружен;
    \item video<N> -- устройство видеозахвата / наложения;
    \item zero -- источник нулевого байта;
    \item zfs -- настраивает пулы хранения ZFS.
\end{enumerate}

На рисунке \ref{root_dir} представлен переход в директорию пользователя root с помощью команды cd root и выведено ее содержимое - команда ls -a.
\addimg{4}{0.95}{Содержимое директории пользователя root}{root_dir}

\newpage
Затем переходим в корневой каталог с помощью команды cd / и откроем файл vmlinuz с помощью тектового редактора vim, выполнив команду vim vmlinuz (рисунок \ref{vmlinuz}).

\addimg{5}{1}{Содержимое файла vmlinuz}{vmlinuz}

\newpage
Чтобы просмотреть права доступа к файлу vmlinuz, выполним команду ls -l, результат выполнения показан на рисунке \ref{vmlinuz_rig}.
\addimg{6}{1}{Содержимое корневого каталога}{vmlinuz_rig}

Как видно из рисунка \ref{vmlinuz_rig} все пользователи и группы пользователей имеют полные права на файл vmlinuz. Владельцем файла указан пользователь root.

Далее добавим нового пользователя user с помощью команды adduser -m user. Результат выполнения команды представлен на рисунке \ref{useradd}.

\addimg{7}{1}{Добавление нового пользователя}{useradd}

\newpage
Создадим 3 файла 1.txt, 2.txt, 3.txt с помощью команд touch, cat и nano. Процесс создания представлен на рисунке \ref{file_cr}.
\addimg{9}{1}{Создание новых файлов различными способами}{file_cr}

Как видно из рисунка \ref{file_cr} владельцем файлов является пользователь root, он имеет полные права на файлы, остальные пользователи имеют права только право на чтение.

Изменим права доступа к файлу 1.txt с помощью команды chmod 777, таким образом все пользователи и группы пользователей получат полные права на файлы 1.txt. Результат выполнения команды представлен на рисунке \ref{chmod}.
\addimg{10}{1}{Изменения прав на файл 1.txt}{chmod}

\newpage
Создадим жесткую и символические ссылки с помощью команд ln и ln -s соответственно. Результат выполненных команд представлен на рисунке \ref{links}.
\addimg{11}{1}{Создание жесткой и символической ссылки на файл 2.txt}{links}

Создадим новую директорию с помощью команды mkdir, результат представлен на рисунке \ref{mkdir}.
\addimg{12}{1}{Создание новой директории}{mkdir}

Для копирования файла 1.txt воспользуемся командой cp (рисунок \ref{cp}). 
\addimg{13}{1}{Копирование файла 1.txt в директорию /new}{cp}

\newpage 
Перемещение файла 2.txt представлено на рисунке \ref{mv}. Чтобы переместить файл была использована команда mv.
\addimg{14}{1}{Перемещение файла 2.txt в директорию /new}{mv}

Изменим владельца файла 3.txt на user с помощью команды chown. Результат выполения команды представлен на рисунке \ref{chown}.
\addimg{16}{1}{Изменение владельца файла 3.txt и директории /new}{chown}

На рисунке \ref{rm} представлена процедура удаление файла 1.txt с помощью команды rm.
\addimg{17}{1}{Удаление файла 1.txt}{rm}

Для того, чтобы удалить непустую дерикторию необходимо воспользоваться командой rm с опцией -R, тогда удалиться директория и все ее содержимое. Результат выполнения представлен на рисунке \ref{rm-R}.
\addimg{18}{1}{Удаление директории /new}{rm-R}

Чтобы найти файл по имени необходимо воспользоваться командой find и после опции -name указать имя искомого файла vga2iso. Результат поиска представлен на рисунке \ref{find}.
\addimg{19}{1}{Выполнение поиска файла vga2iso}{find}