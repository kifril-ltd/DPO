\section*{Контрольные вопросы}
\addcontentsline{toc}{section}{Контрольные вопросы}
1. Что такое файловая система?

Файловая система - это структура, с помощью которой ядро операционной системы предоставляет пользователям (и процессам) ресурсы долговременной памяти системы, т. е. памяти на различного вида долговременных носителях информации  -  жестких дисках, магнитных лентах, CD-ROM и т. п. С точки зрения пользователя, файловая система   —   это логическая структура каталогов и файлов. 

2. Права доступа к файлам. Назначение прав доступа.

В основе механизмов разграничения доступа лежат имена пользователей и имена групп пользователей. В ОС Linux пользователи и группы польщователей могут иметь следующие права на доступ к файлу: чтение из файла, запись в файл и запуск файла на исполнение. Чтобы узнать какие права доступа назначены какому-либо файлу необходимо выполнить команду ls -l <имя файла>. Данная команда выведет следующую информацию о файле: тип файла (файл, каталог, символьная ссылка), информация о правах доступа к файлу владельца, системной группы, для всех остальных польщователей. Чтобы задать новые права для доступа к файлу необходимо воспользоваться командой chmod, которую можно использовать в двух форматах: явное указание кому какие права доступа назначить или отнять и цифровое кодирование прав доступа.

3. Жесткая ссылка в Linux. Основные сведения.

Жёсткая ссылка (HardLink) - это просто ещё одна запись в папке для данного файла.

Когда создаётся жёсткая ссылка, сам файл не копируется физически, а только появляется под ещё одним именем или в ещё одном месте, а его старые имя и местонахождение остаются нетронутыми. С этого момента жёсткая ссылка неотличима от первоначальной записи в папке. Единственное отличие - то, что для жёсткой ссылка не создаётся короткое имя файла, поэтому из ДОС-программ она не видна.

Когда меняется размер или дата файла, все соответствующие записи в папках обновляются автоматически. При удалении файла он не удаляется физически до тех пор, пока все жёсткие ссылки, указывающие на него, не будут удалены. Порядок их удаления значения не имеет. При удалении жёсткой ссылки в корзину количество ссылок у файла сохраняется.


4. Команда поиска в Linux. Основные сведения.

Find - это одна из наиболее важных и часто используемых утилит системы Linux. Это команда для поиска файлов и каталогов на основе специальных условий. Ее можно использовать в различных обстоятельствах, например, для поиска файлов по разрешениям, владельцам, группам, типу, размеру и другим подобным критериям.

Утилита find предустановлена по умолчанию во всех Linux дистрибутивах, поэтому вам не нужно будет устанавливать никаких дополнительных пакетов. Это очень важная находка для тех, кто хочет использовать командную строку наиболее эффективно.

Команда find имеет такой синтаксис:

find [папка] [параметры] критерий шаблон [действие]

5.Перечислите основные команды работы с каталогами.

Основными командами для работы с файлами и каталогами в ОС Linux являются:
\begin{enumerate}
    \item ls -- список файлов в директории;
    \item cd -- переход между директориями;
    \item rm -- удалить файл;
    \item rmdir -- удалить папку;
    \item mv -- переместить файл;
    \item cp -- скопировать файл;
    \item mkdir -- создать папку;
    \item ln -- создать ссылку;
    \item chmod -- изменить права файла;
    \item touch -- создать пустой файл.
\end{enumerate}