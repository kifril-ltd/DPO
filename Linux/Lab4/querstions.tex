\section*{Контрольные вопросы}
\addcontentsline{toc}{section}{Контрольные вопросы}
1. Перечислите состояния задачи в OC Ubuntu

В ОС Ubuntu задачи могут находиться в следующих состояниях:
\begin{itemize}
    \item Running (работа) - процесс работает (он является текущим процессом в системе) или готов к работе (ждет выделения ресурсов процессора);
    \item Sleeping (спячка) - в это состояние процесс переходит при блокировке;
    \item Stopped (остановка) - в этом состоянии процесс останавливает работу, обычно после получения соответствующего сигнала. Например, процесс может быть остановлен для отладки;
    \item Dead (смерть) - в этом состоянии процесс может быть удаден из системы;
    \item Active (активный) и expired (неактивный) - используются при планировании выполнения процесса, и поэтому они не сохраняются в пе-ременной state;
    \item Zombie (зомби) - процесс мертв, то есть он был остановлен, но в системе осталась выполняемая им задача.
\end{itemize}

2. Как создаются задачи в OC Ubuntu?

Задачи создаются путем вызова системной функции clone. Любые обращения к fork или vfork преобразуются в системные вызовы clone во время компиляции. Функция fork создает дочернюю задачу, виртуальная память для кото-рой выделяется по принципу копирования при записи (copy-on-write).Процедура vfork приостанавливает  работу  родительского  процесса  в том случае, когда дочерний процесс вызывает функции execve или exit, чтобы обеспечить загрузку дочерним процессом новых страниц до того, как родительский процесс начнет выполнять бесполезные операции копирования при записи.

3. Назовите классы потоков ОС Ubuntu

В операционной системе Linux алгоритмом диспетчеризации различаются три класса потоков:

\begin{itemize}
    \item Потоки реального времени, обслуживаемые по алго-ритму FIFO;
    \item Потоки  реального  времени,обслуживаемые  в  по-рядке циклической очереди;
    \item Потоки разделения времени.
\end{itemize}

4. Как используется приоритет планирования при запуске задачи

Планировщик различает 40 различных уровней приоритета: от -20 до 19. В  соответствии  с  конвенцией UNIX наименьшее  значение  означает наибольший приоритет в алгоритме планирования (т.е. -20 - это самый высо-кий приоритет. который может иметь процесс). Чем меньше значение приоритета, тем больше ресурсов процессора ей выделяется.

5. Как можно изменить приоритет для выполняющейся задачи?

Чтобы изменить приоритет процесса можно воспользоваться командой renice -n <новый приоритет> <PID процесса>.