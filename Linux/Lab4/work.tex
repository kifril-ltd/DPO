\section{Ход выполнения}
\subsection{Задание 1}

1) Вывести информацию о текущем интерпретаторе команд

Чтобы вывести информацию о текущем интерпретаторе команд необходимо воспользоваться командой echo \$SHELL. Результат выполнения данной команды представлен на рисунке \ref{shell}.

\addimg{1}{1}{Информация о текущем интерпретаторе}{shell}

2) Вывести информацию о текущем пользователе 

Чтобы вывести информацию о текущем пользователе необходимо воспользоваться командой whoami. Результат выполнения данной команды представлен на рисунке \ref{whoami}.

\addimg{2}{1}{Информация о текущем пользователе}{whoami}

3) Вывести информацию о текущем каталоге

Чтобы вывести информацию о текущем каталоге необходимо воспользоваться командой pwd. Результат выполнения данной команды представлен на рисунке \ref{pwd}.

\addimg{3}{1}{Информация о текущем каталоге}{pwd}

4) Вывести информацию об оперативной памяти и области подкачки

Чтобы вывести информацию об оперативной памяти и области подкачки необходимо воспользоваться командой free. Результат выполнения данной команды представлен на рисунке \ref{free}.

\addimg{4}{1}{Информация об оперативной памяти и области подкачки}{free}

5) Вывести информацию о дисковой памяти

Чтобы вывести информацию о дисковой памяти необходимо воспользоваться командой df. Результат выполнения данной команды представлен на рисунке \ref{df}.

\addimg{5}{1}{Информация о дисковой памяти}{df}

\subsection{Задание 2}

1) Получить идентификатор текущего процесса(PID) 

Чтобы получить идентификатор текущего процесса(PID) необходимо воспользоваться командой echo \$\$. Результат выполнения данной команды представлен на рисунке \ref{cr_proc}.

\addimg{6}{1}{Получение идентификатора текущего процесса(PID)}{cr_proc}

2) Получить идентификатор родительского процесса(PPID)  

Чтобы получить идентификатор родительского процесса(PPID)  необходимо воспользоваться командой echo \$PPID. Результат выполнения данной команды представлен на рисунке \ref{ppid}.

\addimg{7}{1}{Получение идентификатора родительского процесса(PPID) }{ppid}

3) Получить идентификатор процесса инициализации системы  

Чтобы получить идентификатор процесса инициализации системы необходимо воспользоваться командой echo pidof init. Результат выполнения данной команды представлен на рисунке \ref{pid_init}.

\addimg{8}{1}{Получение идентификатора процесса инициализации системы}{pid_init}

4) Получить информацию о выполняющихся процессах текущего пользователя в текущем интерпретаторе команд

Чтобы получить информацию о выполняющихся процессах текущего пользователя в текущем интерпретаторе команд необходимо воспользоваться командой ps. Результат выполнения данной команды представлен на рисунке \ref{ps}.

\addimg{9}{1}{Получение информации о выполняющихся процессах текущего пользователя в текущем интерпретаторе команд}{ps}

5) Отобразить все процессы

Чтобы отобразить все процессы необходимо воспользоваться командой ps -e. Результат выполнения данной команды представлен на рисунке \ref{ps_e}.

\addimg{10}{1}{Получение информации всеx процессах}{ps_e}

\subsection{Задание 3}

1) Получить информацию о выполняющихся процессах текущего пользователя в текущем интерпретаторе

Чтобы получить информацию о выполняющихся процессах текущего пользователя в текущем интерпретаторе необходимо воспользоваться командой ps. Результат выполнения данной команды представлен на рисунке \ref{ps_2}.

\addimg{9}{1}{Получение информации о выполняющихся процессах текущего пользователя в текущем интерпретаторе команд}{ps_2}

\newpage
2) Определить текущее значение nice по умолчанию

Чтобы определить текущее значение nice по умолчанию необходимо воспользоваться командой nice. Результат выполнения данной команды представлен на рисунке \ref{nice}.

\addimg{11}{1}{Получение текущего значения nice по умолчанию}{nice}

3) Запустить интерпретатор bash с понижением приоритета

Чтобы запустить интерпретатор bash с понижением приоритета необходимо воспользоваться командой nice -n 10 bash. Результат выполнения данной команды представлен на рисунке \ref{nice_n}.

\addimg{12}{1}{Запуск интерпретатора bash с понижением приоритета}{nice_n}

4) Определить PID запущенного интерпретатора

Чтобы определить PID запущенного интерпретатора необходимо воспользоваться командой ps -f. Результат выполнения данной команды представлен на рисунке \ref{ps_f}.

\addimg{13}{1}{Определение PID запущенного интерпретатора}{ps_f}

Таким образом PID запущенного интерпретатора равен 23143.

5) Установить приоритет запущенного интерпретатора равным 5

Чтобы установить приоритет запущенного интерпретатора равным 5 необходимо воспользоваться командой renice –n  5 <PID процесса>. Результат выполнения данной команды представлен на рисунке \ref{renice}.

\addimg{14}{1}{Установка приоритета запущенного интерпретатора равным 5}{renice}

6) Получить информацию о процессах bash

Чтобы получить информацию о процессах bash необходимо воспользоваться командой ps lax | grep bash. Результат выполнения данной команды представлен на рисунке \ref{ps_lax}.

\addimg{15}{1}{Получение информации о процессах bash}{ps_lax}
