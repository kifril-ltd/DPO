\section*{Контрольные вопросы}
\addcontentsline{toc}{section}{Контрольные вопросы}

1. Что такое ключ ssh? В чем преимущество их использования?

SSH-ключи  используются  для  идентификации  клиента  приподключении к удалённому серверу. SSH-ключи представляют собой паруключей –приватный и публичный. Приватный ключ хранится в закрытомдоступе  у  клиента,  публичный  отправляется  на  сервер.Преимущество использования ключей в удобстве (не нужно запоминатьпароли) и безопасности (взломать приватный ssh-ключ достаточно сложно).

2. Как сгенерировать ключи ssh в разных ОС?

Генерация ssh-ключа в ОС Linux возможна с помощью команды sshkeygen.В ОС  Windows  можно  использовать  программу  PuTTY  для  генерацииssh-ключей и подключения по shh-протоколу.

3. Возможно ли из «секретного» ключасгенерировать «публичный»и/или наоборот?

Нет, невозможно.

4. Будут ли отличаться пары ключей, сгенерированные на одном ПК несколько  раз  с  исходными  условиями  (наличие/отсутствие  пароля  на«секретный» ключ и т.п.)

Да, будут. Утилита ssh-keygen каждый раз случайно генерирует пару ключей.

5. Перечислите доступные ключи для ssh-keygen.exe

\begin{itemize}
    \item DSA;
    \item RSA;
    \item ECDASA;
    \item Ed25519.
\end{itemize}

6. Можно ли использовать один «секретный» ключ доступа с разных ОС, установленных на одном ПК/на разных ПК?

Можно, но безопасность такого ключа уже не гарантирована.

7. Возможно ли организовать подключение «по ключу» ssh к системеc ОС Windows, в которой запущен OpenSSH сервер?

Да, возможно, с использованием программы PuTTY.

8. Какие известные Вам сервисы сети Интернет позволяюторганизовать доступ к ресурсам посредством SSH ключей?

GitHub