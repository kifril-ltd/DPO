\section*{Контрольные вопросы}
\addcontentsline{toc}{section}{Контрольные вопросы}

1. Назовите  отличия  использования  контейнеров  по  сравнению  с виртуализацией. 

A. Меньшие накладные расходы на инфраструктуру

2. Назовите основные компоненты Docker. 

B. Контейнеры

3. Какие технологии используются для работы с контейнерами?

C. Контрольные группы (cgroups)

4. Найдите соответствие между компонентом и его описанием:
\begin{itemize}
    \item образы -- доступные только для чтения шаблоны приложений;
    \item контейнеры -- изолированные при помощи технологий операционной системы пользовательские окружения, в которых выполняются приложения;
    \item реестры (репозитории) – сетевые хранилища образов.
\end{itemize}

5. В чем отличие контейнеров от виртуализации?

Виртуальная машина – программная и/или аппаратная система, эмулирующая аппаратное обеспечение некоторой целевой и исполняющая программы для гостевой платформы на платформе-хозяине (хосте) или виртуализирующая некоторую платформу и создающая на ней среды, изолирующие друг от друга программы и даже операционные системы. Виртуальные машины запускают на физических машинах, используя гипервизор.

В отличие от виртуальной машины, обеспечивающей аппаратную виртуализацию, контейнер обеспечивает виртуализацию на уровне операционной системы с помощью абстрагирования пользовательского пространства.

В целом контейнеры выглядят как виртуальные машины. Например, у них есть изолированное пространство для запуска приложений, они позволяют выполнять команды с правами суперпользователя, имеют частный сетевой интерфейс и IP-адрес, пользовательские маршруты и правила межсетевого экрана и т. д.

Одна большая разница между контейнерами и виртуальными машинами в том, что контейнеры разделяют ядро хоста с другими контейнерами.

6.Перечислите основные команды утилиты Docker с их кратким описанием.

\begin{itemize}
    \item docker ps  — показывает список запущенных контейнеров;
    \item docker pull  —  скачать определённый образ или набор образов (репозиторий);
    \item docker build  —  эта команда собирает образ Docker из Dockerfile и «контекста»;
    \item docker run  —  запускает контейнер, на основе указанного образа;
    \item docker logs  —  эта команда используется для просмотра логов указанного контейнера;
    \item docker volume ls  —  показывает список томов, которые являются предпочитаемым механизмом для сохранения данных, генерируемых и используемых контейнерами Docker;
    \item docker rm  —  удаляет один и более контейнеров;
    \item docker rmi  —  удаляет один и более образов;
    \item docker stop  —  останавливает один и более контейнеров;
    \item docker exec –it ... - выполняет команду в определенном контейнере
\end{itemize}

7. Каким образом осуществляется поиск образов контейнеров?

Сначала проверяется локальный репозиторий на наличия нужного контейнера, если он не найден локально, то поиск производится в репозитории Docker Hub.

8. Каким образом осуществляется запуск контейнера?

Для запуска контейнера его необходимо изначально создать из образа, поэтому изначально контейнер собирается с помощью команды docker build,а уже затем запускается с помощью команды docker run.

9. Что значит управлять состоянием контейнеров?

Это означает, что в любой момент времени есть возможность запустить, остановить или выполнить команды внутри контейнера.

10. Как изолировать контейнер?

Контейнеры уже по сути своей являются изолированными единицами, поэтому достаточно без ошибок сконфигурировать файлы Dockerfile и/или docker-compose.yml.

11. Опишите последовательность создания новых образов, назначение Dockerfile?

Производится выбор основы для нового образа на Docker Hub, далее производится конфигурация Dockerfile, где описываются все необходимые пакеты, файлы, команды и т.п.

Dockerfile — это текстовый файл с инструкциями, необходимыми для создания образа контейнера. Эти инструкции включают идентификацию существующего образа, используемого в качестве основы, команды, выполняемые в процессе создания образа, и команду, которая будет выполняться при развертывании новых экземпляров этого образа контейнера.

12. Возможно ли работать с контейнерами Docker без одноименного движка?

Да, если использовать Kubernetes

13. Опишите назначение системы оркестрации контейнеров Kubernetes. Перечислите основные объекты Kubernetes?

Kubernetes — открытое программное обеспечение для автоматизации развёртывания, масштабирования контейнеризированных приложений и управления ими. Поддерживает основные технологии контейнеризации, включая Docker, rkt, также возможна поддержка технологий аппаратной виртуализации.

\begin{itemize}
    \item Nodes: Нода это машина в кластере Kubernetes.
    \item Pods: Pod это группа контейнеров с общими разделами, запускаемых как единое целое.
    \item Replication Controllers: replication controller гарантирует, что определенное количество «реплик» pod'ы будут запущены в любой момент времени.
    \item Services: Сервис в Kubernetes это абстракция которая определяет логический объединённый набор pod и политику доступа к ним.
    \item Volumes: Volume(раздел) это директория, возможно, с данными в ней, которая доступна в контейнере.
    \item Labels: Label'ы это пары ключ/значение которые прикрепляются к объектам, например pod'ам. Label'ы могут быть использованы для создания и выбора наборов объектов.
    \item Kubectl Command Line Interface: kubectl интерфейс командной строки для управления Kubernetes.
\end{itemize}