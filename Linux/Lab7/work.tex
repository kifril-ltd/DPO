\section{Ход работы}

\subsection{Подключение к удаленному серверу по паролю}

Для того, чтобы авторизоваться на сервере с помощью выданного логина и пароля воспользуемся командой ssh -l <логин>. После запуска данной команды система потребует пароль и затем даст доступ к серверу. Результат выполнения приведенных действий проиллюстрирован на рисунке \ref{ssh_pass}.

\addimg{1}{1}{Подключение к удаленному серверу по логину и паролю}{ssh_pass}

\subsection{Просмотр окружения пользователя}

После успешного подключения к удаленному серверу мы можем проверить окружение пользователя с помощью стандартной команды ls -al. Результат выполнения команды представлен на рисунке \ref{ssh_ls}.

\addimg{2}{1}{Проверка окружения пользователя}{ssh_ls}

\subsection{Генерация пары ключей доступа к серверу}

Для генерации ключей используем команду ssh-genkey. После выполнения данной команды сгенерируется пара ключей: приватный id\_rsa и публичный id\_rsa.pub. Результат выполнения данной команды представлен на рисунке \ref{ssh_keygen}.

\addimg{3}{0.9}{Генерация пары ключей досупа к серверу}{ssh_keygen}

\subsection{Передача публичного ключа на сервер}

Для передачи публичного ключа на сервер воспользуемся командой ssh-copy-id -i <путь до ключа>. Результат выполнения команды представлен на рисунке \ref{ssh_copy_id}.

\addimg{4}{1}{Передача публичного ключа на сервер}{ssh_copy_id}

Проверим подключение к серверу без использования пароля. Результат подключения представлен на рисунке \ref{ssh_login}.

\addimg{5}{1}{Подключение к удаленному серверу без  использования пароля}{ssh_login}

\subsection{Организация подключения к серверу по имени}

Для подключения к серверу по имени, необходимо создать файл конфигурации в каталоге .ssh со следующим содержанием:

\addimg{6}{1}{Файл конфигурации}{}

Проверим подключение к серверу по указанному имени. Результат подключения представлен на рисунке \ref{ssh_name_login}.

\addimg{7}{1}{Подключение к серверу по имени}{ssh_name_login}